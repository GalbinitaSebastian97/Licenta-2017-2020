\documentclass[11pt,a4paper]{report}
\usepackage{fancyhdr}
\usepackage{amssymb}
\usepackage{amsmath}
\usepackage{graphicx}
\usepackage[T1]{fontenc}
\usepackage{hyperref}
\hypersetup{colorlinks=true,linkcolor=cyan, citecolor=green,filecolor=black,urlcolor=blue}
\usepackage{epstopdf}
\usepackage{makeidx}
\usepackage{tocloft}
\renewcommand{\cftchapleader}{\cftdotfill{\cftdotsep}}
\renewcommand{\contentsname}{Cuprins}
\renewcommand{\bibname}{B\lowercase{ibliografie}}
\renewcommand{\chaptername}{Capitol}
\renewcommand{\appendixname}{Anexa}
\renewcommand{\indexname}{I\lowercase{indice}}
\pagestyle{fancy}
\makeindex
\newtheorem{prop}{Propozi\c tie}
\newenvironment{demo}{\paragraph{\textbf{Demonstra\c{t}ie}:}}{\hfill$\square$}
\newcommand{\R}{\mathbb{R}}
\begin{document}
\begin{center}
\begin{tabular}{ |p{2.9cm}|p{2cm}|p{6cm}|  }
	
 \hline
 \multicolumn{3}{|c|}{MiniGPS} \\
 \hline

  Variabile & Tipul & Descriere\\
 \hline

 $bNGraph$        & $CButton$   & Buton pentru generarea unui nou graf\vspace{0.1cm}  \\
 \hline
 $BCompute$       & $CButton$   & Compune cel mai scurt drum\vspace{0.1cm} \\
 \hline
 $Autobuz[i].path[k]$ & $int$       & Drumul $k$ \^ in autobuzul $i$\vspace{0.1cm} \\
 \hline
 $Autobuz[i].nNod$  & $int$       & Num\u arul de noduri \^ in autobuzul $i$ \vspace{0.1cm}\\
 \hline
 $Autobuz[i].sp$      & $int$       & Costul total al autobuzului $i$\vspace{0.1cm}\\
 \hline
 $home$           & $CPonit$    & Col\c tul din st\^ anga sus al zonei grilei dreptunghiului \vspace{0.1cm}   \\
 \hline
 $v[i].cost[j]$     & $int$       & Costul dintre $(v_{i}, v_{j})$\vspace{0.1cm}\\
 \hline
 $v[i].sp[j]$     & $int$       & Drumul cel mai scurt dintre $(v_{i}, v_{j})$\vspace{0.1cm}\\
 \hline
 $Pv$             & $int$       & Nodul precedent al nodului curent\vspace{0.1cm}\\
 \hline
 $Sursa, \newline Destinatie$         & $int$       & Sursa \c si destina\c tia\vspace{0.1cm}\\
 \hline
 $nBus$           & $int$       & Numarul de autobuze de succes\vspace{0.1cm}\\
 \hline
 $table$          & $CListCtrl$ & Tabela care afi\c seaz\u a autobuzele de succes\vspace{0.1cm}\\
 \hline
 $pBus[i]$        & $CPen$      & Culoarea autobuzului i\vspace{0.1cm}\\
 \hline
 $N$              & $constant$  & Num\u arul de noduri \^ in graf\vspace{0.1cm}\\
 \hline
 $LinkRange$      & $constant$  & Valoarea pragului intervalului pentru adiacen\c ta dintre dou\u anoduri din grafic\vspace{0.1cm}\\
 \hline
 $fv[i]$          & $bool$      & Starea v\^ arfului $v_{i}$\vspace{0.1cm}\\
 
 \hline
\end{tabular}

\end{center}
\end{document}